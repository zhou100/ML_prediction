\documentclass[]{elsarticle} %review=doublespace preprint=single 5p=2 column
%%% Begin My package additions %%%%%%%%%%%%%%%%%%%
\usepackage[hyphens]{url}
\usepackage{lineno} % add
\providecommand{\tightlist}{%
  \setlength{\itemsep}{0pt}\setlength{\parskip}{0pt}}

\bibliographystyle{elsarticle-harv}
\biboptions{sort&compress} % For natbib
\usepackage{graphicx}
\usepackage{booktabs} % book-quality tables
%% Redefines the elsarticle footer
%\makeatletter
%\def\ps@pprintTitle{%
% \let\@oddhead\@empty
% \let\@evenhead\@empty
% \def\@oddfoot{\it \hfill\today}%
% \let\@evenfoot\@oddfoot}
%\makeatother

% A modified page layout
\textwidth 6.75in
\oddsidemargin -0.15in
\evensidemargin -0.15in
\textheight 9in
\topmargin -0.5in
%%%%%%%%%%%%%%%% end my additions to header

\usepackage[T1]{fontenc}
\usepackage{lmodern}
\usepackage{amssymb,amsmath}
\usepackage{ifxetex,ifluatex}
\usepackage{fixltx2e} % provides \textsubscript
% use upquote if available, for straight quotes in verbatim environments
\IfFileExists{upquote.sty}{\usepackage{upquote}}{}
\ifnum 0\ifxetex 1\fi\ifluatex 1\fi=0 % if pdftex
  \usepackage[utf8]{inputenc}
\else % if luatex or xelatex
  \usepackage{fontspec}
  \ifxetex
    \usepackage{xltxtra,xunicode}
  \fi
  \defaultfontfeatures{Mapping=tex-text,Scale=MatchLowercase}
  \newcommand{\euro}{€}
\fi
% use microtype if available
\IfFileExists{microtype.sty}{\usepackage{microtype}}{}
\usepackage{color}
\usepackage{fancyvrb}
\newcommand{\VerbBar}{|}
\newcommand{\VERB}{\Verb[commandchars=\\\{\}]}
\DefineVerbatimEnvironment{Highlighting}{Verbatim}{commandchars=\\\{\}}
% Add ',fontsize=\small' for more characters per line
\usepackage{framed}
\definecolor{shadecolor}{RGB}{248,248,248}
\newenvironment{Shaded}{\begin{snugshade}}{\end{snugshade}}
\newcommand{\KeywordTok}[1]{\textcolor[rgb]{0.13,0.29,0.53}{\textbf{#1}}}
\newcommand{\DataTypeTok}[1]{\textcolor[rgb]{0.13,0.29,0.53}{#1}}
\newcommand{\DecValTok}[1]{\textcolor[rgb]{0.00,0.00,0.81}{#1}}
\newcommand{\BaseNTok}[1]{\textcolor[rgb]{0.00,0.00,0.81}{#1}}
\newcommand{\FloatTok}[1]{\textcolor[rgb]{0.00,0.00,0.81}{#1}}
\newcommand{\ConstantTok}[1]{\textcolor[rgb]{0.00,0.00,0.00}{#1}}
\newcommand{\CharTok}[1]{\textcolor[rgb]{0.31,0.60,0.02}{#1}}
\newcommand{\SpecialCharTok}[1]{\textcolor[rgb]{0.00,0.00,0.00}{#1}}
\newcommand{\StringTok}[1]{\textcolor[rgb]{0.31,0.60,0.02}{#1}}
\newcommand{\VerbatimStringTok}[1]{\textcolor[rgb]{0.31,0.60,0.02}{#1}}
\newcommand{\SpecialStringTok}[1]{\textcolor[rgb]{0.31,0.60,0.02}{#1}}
\newcommand{\ImportTok}[1]{#1}
\newcommand{\CommentTok}[1]{\textcolor[rgb]{0.56,0.35,0.01}{\textit{#1}}}
\newcommand{\DocumentationTok}[1]{\textcolor[rgb]{0.56,0.35,0.01}{\textbf{\textit{#1}}}}
\newcommand{\AnnotationTok}[1]{\textcolor[rgb]{0.56,0.35,0.01}{\textbf{\textit{#1}}}}
\newcommand{\CommentVarTok}[1]{\textcolor[rgb]{0.56,0.35,0.01}{\textbf{\textit{#1}}}}
\newcommand{\OtherTok}[1]{\textcolor[rgb]{0.56,0.35,0.01}{#1}}
\newcommand{\FunctionTok}[1]{\textcolor[rgb]{0.00,0.00,0.00}{#1}}
\newcommand{\VariableTok}[1]{\textcolor[rgb]{0.00,0.00,0.00}{#1}}
\newcommand{\ControlFlowTok}[1]{\textcolor[rgb]{0.13,0.29,0.53}{\textbf{#1}}}
\newcommand{\OperatorTok}[1]{\textcolor[rgb]{0.81,0.36,0.00}{\textbf{#1}}}
\newcommand{\BuiltInTok}[1]{#1}
\newcommand{\ExtensionTok}[1]{#1}
\newcommand{\PreprocessorTok}[1]{\textcolor[rgb]{0.56,0.35,0.01}{\textit{#1}}}
\newcommand{\AttributeTok}[1]{\textcolor[rgb]{0.77,0.63,0.00}{#1}}
\newcommand{\RegionMarkerTok}[1]{#1}
\newcommand{\InformationTok}[1]{\textcolor[rgb]{0.56,0.35,0.01}{\textbf{\textit{#1}}}}
\newcommand{\WarningTok}[1]{\textcolor[rgb]{0.56,0.35,0.01}{\textbf{\textit{#1}}}}
\newcommand{\AlertTok}[1]{\textcolor[rgb]{0.94,0.16,0.16}{#1}}
\newcommand{\ErrorTok}[1]{\textcolor[rgb]{0.64,0.00,0.00}{\textbf{#1}}}
\newcommand{\NormalTok}[1]{#1}
\usepackage{graphicx}
% We will generate all images so they have a width \maxwidth. This means
% that they will get their normal width if they fit onto the page, but
% are scaled down if they would overflow the margins.
\makeatletter
\def\maxwidth{\ifdim\Gin@nat@width>\linewidth\linewidth
\else\Gin@nat@width\fi}
\makeatother
\let\Oldincludegraphics\includegraphics
\renewcommand{\includegraphics}[1]{\Oldincludegraphics[width=\maxwidth]{#1}}
\ifxetex
  \usepackage[setpagesize=false, % page size defined by xetex
              unicode=false, % unicode breaks when used with xetex
              xetex]{hyperref}
\else
  \usepackage[unicode=true]{hyperref}
\fi
\hypersetup{breaklinks=true,
            bookmarks=true,
            pdfauthor={},
            pdftitle={Predicting Food Insecurity with Machine Learning based on readily avaliable data},
            colorlinks=true,
            urlcolor=blue,
            linkcolor=magenta,
            pdfborder={0 0 0}}
\urlstyle{same}  % don't use monospace font for urls
\setlength{\parindent}{0pt}
\setlength{\parskip}{6pt plus 2pt minus 1pt}
\setlength{\emergencystretch}{3em}  % prevent overfull lines
\setcounter{secnumdepth}{0}
% Pandoc toggle for numbering sections (defaults to be off)
\setcounter{secnumdepth}{0}
% Pandoc header


\usepackage[nomarkers]{endfloat}

\begin{document}
\begin{frontmatter}

  \title{Predicting Food Insecurity with Machine Learning based on readily
avaliable data}
    \author[University Of Illinois At Urbana-Champaign]{Yujun Zhou\corref{c1}}
   \ead{zhou100@illinois.edu} 
   \cortext[c1]{Corresponding Author}
      \address[University Of Illinois At Urbana-Champaign]{Department of Agriculture and Consumer Economics, University of
Illinois, USA}
  
  \begin{abstract}
  We determine whether corn and soybean futures contract prices are
  stationary or not.
  \end{abstract}
   \begin{keyword} prices, unit root, stationarity \sep \end{keyword}
 \end{frontmatter}

\begin{Shaded}
\begin{Highlighting}[]
\CommentTok{# Seed for random number generation}
\KeywordTok{set.seed}\NormalTok{(}\DecValTok{42}\NormalTok{)}
\end{Highlighting}
\end{Shaded}

Temporal Poverty Prediction using Satellite Imagery(Chen 2017).

\begin{enumerate}
\def\labelenumi{\arabic{enumi}.}
\tightlist
\item
  measuring and tracking areas of poor is a necessary step for targeting
  aid and guiding policy decisions.
\item
  obtaining data is time and labor intensive
\item
  (Blumenstock 2016)
\end{enumerate}

Import the Malawi data.

\begin{Shaded}
\begin{Highlighting}[]
\CommentTok{# import the data }
\NormalTok{ihs2010<-}\KeywordTok{read.csv}\NormalTok{(}\StringTok{"data/cleaned/Malawi/IHS2010.csv"}\NormalTok{)}
\NormalTok{ihs2013<-}\KeywordTok{read.csv}\NormalTok{(}\StringTok{"data/cleaned/Malawi/IHS2013.csv"}\NormalTok{)}
\end{Highlighting}
\end{Shaded}

Organize the variable names and ready for analysis.

\begin{Shaded}
\begin{Highlighting}[]
\CommentTok{# levels }
\NormalTok{levels<-}\KeywordTok{c}\NormalTok{(}\StringTok{"ipczone"}\NormalTok{,}\StringTok{"TA"}\NormalTok{,}\StringTok{"clust"}\NormalTok{)}

\CommentTok{# variables }
\NormalTok{weather<-}\KeywordTok{c}\NormalTok{(}\StringTok{"L12raincytot"}\NormalTok{,}\StringTok{"L12day1rain"}\NormalTok{,}\StringTok{"L12maxdays"}\NormalTok{,}\StringTok{"floodmax"}\NormalTok{)}
\NormalTok{access<-}\KeywordTok{c}\NormalTok{(}\StringTok{"lag_price"}\NormalTok{,}\StringTok{"lag_thinn"}\NormalTok{)}
\NormalTok{asset1 <-}\KeywordTok{c}\NormalTok{(}\StringTok{"roof"}\NormalTok{,}\StringTok{"cells_own"}\NormalTok{)}
\NormalTok{land<-}\KeywordTok{c}\NormalTok{(}\StringTok{"percent_ag"}\NormalTok{,}\StringTok{"elevation"}\NormalTok{,}\StringTok{"nutri_reten_constrained"}\NormalTok{)}
\NormalTok{distance<-}\KeywordTok{c}\NormalTok{(}\StringTok{"dist_road"}\NormalTok{,}\StringTok{"dist_admarc"}\NormalTok{)}
\NormalTok{demo<-}\KeywordTok{c}\NormalTok{(}\StringTok{"hhsize"}\NormalTok{,}\StringTok{"hh_age"}\NormalTok{,}\StringTok{"hh_gender"}\NormalTok{,}\StringTok{"asset"}\NormalTok{)}

\NormalTok{model3_variables<-}\KeywordTok{c}\NormalTok{(weather,access,asset1,land,distance,demo)}
\NormalTok{model2_variables<-}\KeywordTok{c}\NormalTok{(weather,access,asset1,land,distance)}
\NormalTok{model1_variables<-}\KeywordTok{c}\NormalTok{(weather,access,land,distance)}

\CommentTok{# goal : combine variables at different levels using pastes }
\CommentTok{# output: variables lists at different levels, TA_vars, ipczone vars, etc. }

\ControlFlowTok{for}\NormalTok{ (level }\ControlFlowTok{in}\NormalTok{ levels)\{}
  \CommentTok{# assign levels of variables group}
\NormalTok{  group_var_name<-}\KeywordTok{paste}\NormalTok{(level,}\StringTok{"vars"}\NormalTok{,}\DataTypeTok{sep=}\StringTok{"_"}\NormalTok{)}
  \KeywordTok{assign}\NormalTok{(group_var_name,}\KeywordTok{c}\NormalTok{())}
  
  \ControlFlowTok{for}\NormalTok{(var }\ControlFlowTok{in}\NormalTok{ model3_variables)\{}
\NormalTok{    temp<-}\KeywordTok{paste}\NormalTok{(level,var,}\DataTypeTok{sep =} \StringTok{"_"}\NormalTok{)}
\NormalTok{    new<-}\KeywordTok{append}\NormalTok{(}\KeywordTok{get}\NormalTok{(group_var_name),temp)}
    \KeywordTok{assign}\NormalTok{(group_var_name,new)}
\NormalTok{  \}}
\NormalTok{\}}
\end{Highlighting}
\end{Shaded}

\subsubsection{1. Linear/tobit Results}\label{lineartobit-results}

Create the formulas using the formula\_compose function.

\begin{Shaded}
\begin{Highlighting}[]
\NormalTok{rcsi_formula<-}\KeywordTok{formula_compose}\NormalTok{(}\StringTok{"RCSI"}\NormalTok{,clust_vars)}
\NormalTok{logFCS_formula<-}\KeywordTok{formula_compose}\NormalTok{(}\StringTok{"logFCS"}\NormalTok{,clust_vars)}
\NormalTok{HDDS_formula<-}\KeywordTok{formula_compose}\NormalTok{(}\StringTok{"HDDS"}\NormalTok{,clust_vars)}
\end{Highlighting}
\end{Shaded}

\begin{Shaded}
\begin{Highlighting}[]
\NormalTok{rcsi_predictions<-}\KeywordTok{linear_fit}\NormalTok{(rcsi_formula,ihs2010,ihs2013)}
\CommentTok{# lm_train_measure<-postResample(rcsi_predictions$pred_train,ihs2010$RCSI)}
\NormalTok{lm_test_measure<-}\KeywordTok{postResample}\NormalTok{(rcsi_predictions}\OperatorTok{$}\NormalTok{pred_test,ihs2013}\OperatorTok{$}\NormalTok{RCSI)}
\NormalTok{lm_test_measure}
\end{Highlighting}
\end{Shaded}

\begin{verbatim}
##       RMSE   Rsquared        MAE 
## 7.07641265 0.01254163 5.52598695
\end{verbatim}

\begin{Shaded}
\begin{Highlighting}[]
\CommentTok{# scatter.smooth(rcsi_predictions$pred_test,ihs2013$RCSI)}
\end{Highlighting}
\end{Shaded}

try tobit instead for RCSI. The prediction value (unconditional mean)
should actually be different with the assumption of non-normal /
Gaussian error. The old predication function returns a latent mean.

\begin{Shaded}
\begin{Highlighting}[]
\NormalTok{tobit_rcsi<-}\KeywordTok{tobit}\NormalTok{(rcsi_formula,}\DataTypeTok{left =} \DecValTok{0}\NormalTok{,}\DataTypeTok{right =} \OtherTok{Inf}\NormalTok{,}\DataTypeTok{data =}\NormalTok{ihs2010)}
\NormalTok{mu <-}\StringTok{ }\KeywordTok{predict}\NormalTok{(tobit_rcsi,}\DataTypeTok{newdata=}\NormalTok{ ihs2013)}
\NormalTok{sigma <-}\StringTok{ }\NormalTok{tobit_rcsi}\OperatorTok{$}\NormalTok{scale}
\NormalTok{p0 <-}\StringTok{ }\KeywordTok{pnorm}\NormalTok{(mu}\OperatorTok{/}\NormalTok{sigma)}
\NormalTok{lambda <-}\StringTok{ }\ControlFlowTok{function}\NormalTok{(x) }\KeywordTok{dnorm}\NormalTok{(x)}\OperatorTok{/}\KeywordTok{pnorm}\NormalTok{(x)}
\NormalTok{ey0 <-}\StringTok{ }\NormalTok{mu }\OperatorTok{+}\StringTok{ }\NormalTok{sigma }\OperatorTok{*}\StringTok{ }\KeywordTok{lambda}\NormalTok{(mu}\OperatorTok{/}\NormalTok{sigma)}
\NormalTok{ey <-}\StringTok{ }\NormalTok{p0 }\OperatorTok{*}\StringTok{ }\NormalTok{ey0}
\NormalTok{RCSI_tobit_prediction<-ey}
\end{Highlighting}
\end{Shaded}

There are some improvement but still slighlty

\begin{Shaded}
\begin{Highlighting}[]
\NormalTok{lm_test_measure<-}\KeywordTok{postResample}\NormalTok{(RCSI_tobit_prediction,ihs2013}\OperatorTok{$}\NormalTok{RCSI)}
\NormalTok{lm_test_measure}
\end{Highlighting}
\end{Shaded}

\begin{verbatim}
##       RMSE   Rsquared        MAE 
## 7.06342825 0.01313111 5.40365933
\end{verbatim}

\begin{Shaded}
\begin{Highlighting}[]
\KeywordTok{scatter.smooth}\NormalTok{(RCSI_tobit_prediction,ihs2013}\OperatorTok{$}\NormalTok{RCSI)}
\end{Highlighting}
\end{Shaded}

\includegraphics{manuscript_files/figure-latex/unnamed-chunk-8-1.pdf}

\begin{Shaded}
\begin{Highlighting}[]
\NormalTok{logFCS_predictions<-}\KeywordTok{linear_fit}\NormalTok{(logFCS_formula,ihs2010,ihs2013)}
\NormalTok{lm_test_measure<-}\KeywordTok{postResample}\NormalTok{(rcsi_predictions}\OperatorTok{$}\NormalTok{pred_test,ihs2013}\OperatorTok{$}\NormalTok{logFCS)}
\NormalTok{lm_test_measure}
\end{Highlighting}
\end{Shaded}

\begin{verbatim}
##       RMSE   Rsquared        MAE 
## 2.47188996 0.09126124 2.11332725
\end{verbatim}

\begin{Shaded}
\begin{Highlighting}[]
\CommentTok{# scatter.smooth(logFCS_predictions$pred_test,ihs2013$logFCS)}
\end{Highlighting}
\end{Shaded}

\begin{Shaded}
\begin{Highlighting}[]
\NormalTok{HDDS_predictions<-}\KeywordTok{linear_fit}\NormalTok{(HDDS_formula,ihs2010,ihs2013)}
\NormalTok{lm_test_measure<-}\KeywordTok{postResample}\NormalTok{(rcsi_predictions}\OperatorTok{$}\NormalTok{pred_test,ihs2013}\OperatorTok{$}\NormalTok{HDDS)}
\NormalTok{lm_test_measure}
\end{Highlighting}
\end{Shaded}

\begin{verbatim}
##       RMSE   Rsquared        MAE 
## 2.38812209 0.08306447 1.84598055
\end{verbatim}

\begin{Shaded}
\begin{Highlighting}[]
\CommentTok{# scatter.smooth(HDDS_predictions$pred_test,ihs2013$HDDS)}
\end{Highlighting}
\end{Shaded}

\section{Methods}\label{methods}

\subsection{Participants}\label{participants}

\subsection{Material}\label{material}

\subsection{Procedure}\label{procedure}

\subsection{Data analysis}\label{data-analysis}

We used for all our analyses.

\section{Results}\label{results}

\section{Discussion}\label{discussion}

\newpage

\section*{References}\label{references}
\addcontentsline{toc}{section}{References}

\hypertarget{refs}{}
\hypertarget{ref-blumenstock2016fighting}{}
Blumenstock, Joshua Evan. 2016. ``Fighting Poverty with Data.''
\emph{Science} 353 (6301). American Association for the Advancement of
Science: 753--54.

\hypertarget{ref-chentemporal}{}
Chen, Derek. 2017. ``Temporal Poverty Prediction Using Satellite
Imagery.''

\end{document}


